\section{Introduction}
\label{cha:intro}

% BACKGROUND AND MOTIVATION
Cloud computing delivers on-demand resources like servers, storage, and databases via the Internet, eliminating the need for physical infrastructure and enabling flexibility and scalability. Its adoption has surged in the past decade and Amazon Web Services (AWS), has emerged as a leader and pioneer of various cloud models. AWS maintains 31\% market share driven by its innovation and broad service portfolio \cite{web_richter_cloud_market}.

% PROBLEM STATEMENT
Applications in cloud environments typically consist of compute and datastore service instances that interact frequently, therefore latency between these services significantly influence overall performance and user experience of the application \cite{atricle_dean_tail,book_popescu_netlat}. AWS provides performance insights for its datastore services from both the service-side and client-side perspectives. The service-side perspective evaluates how efficiently the datastore performs internally, while the client-side perspective focuses on the time taken for a client to receive a response after submitting a query. However, it remains unclear how client-side latency varies across different AWS compute services.

% THESIS CONTRIBUTION
Prominent AWS compute services include (1) Elastic Cloud Compute (EC2), offers customizable virtual machines for consistent performance and long-running workloads, and (2) Lambda, a serverless, event-driven service ideal for short tasks and automatic scaling. Key datastore services are (1) Relational Database Service (RDS), provides managed SQL databases for structured data, (2) DynamoDB, a serverless NoSQL database for high-scale, low-latency use cases, and (3) Simple Storage Service (S3), a distributed object storage service for unstructured data.

In this thesis, we benchmark access latency between the above AWS compute and datastore services to examine how the choice of compute service affects latency dynamics with datastores, such as whether an EC2-RDS pair outperforms a Lambda-RDS pair. The results consistently show that EC2-based pairs outperform Lambda-based pairs by small margins. This thesis provides insights for cloud architects and application developers to optimize AWS compute and datastore service selection based on access latency, supporting decision-making for latency-sensitive applications and improving cloud architecture and cost-efficiency.

We therefore make the following contributions in this thesis:
\begin{enumerate}
	\item We address the identified knowledge gap by proposing a benchmark in \cref{cha:approach}.
	\item We evaluate the proposed approach through experimentation in \cref{cha:implementation}.
	\item We present and interpret our results in \cref{cha:results}.
	\item We discuss our observations, possible causes, and limitations in \cref{cha:discuss}.
\end{enumerate}
