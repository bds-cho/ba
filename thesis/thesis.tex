% This thesis template is based on the personal template of Felix Moebius. Thanks, Felix!
% It has been extended and modified by Tobias Pfandzelter at the Scalable Software Systems group of TU Berlin.
% It is licensed under the terms of the MIT license, meaning you are free to use it however you see fit but we accept no liability.
% Good luck writing your thesis!

\documentclass[a4paper, 11pt]{article}

\usepackage[utf8x]{inputenc}
\usepackage[T1]{fontenc}
\usepackage{fix-cm}

\usepackage[a4paper, margin=3cm]{geometry}
\usepackage[titletoc, title]{appendix}

\usepackage{color}
\usepackage{booktabs}
\usepackage[all]{nowidow}
\usepackage[dvipsnames]{xcolor}
\usepackage[hidelinks]{hyperref}
\usepackage{acronym}
\usepackage{graphicx}
\usepackage{url}
\usepackage{titlesec}
\usepackage{csquotes}

\usepackage{transparent}
\usepackage{eso-pic}
\usepackage[section]{placeins}
\usepackage{setspace}
\usepackage{parskip}
\usepackage{subcaption}
\usepackage{color,soul}

\renewcommand\thefigure{%
\thesection.\arabic{figure}}
\renewcommand\thesubfigure{%
\thesection.\arabic{figure}.\arabic{subfigure}}
\renewcommand\thetable{%
\thesection.\arabic{table}}

\usepackage[main=english, ngerman]{babel}

% we use the cleveref package to refer to figures, sections, etc.
% instead of "Figure~\ref{fig:example}", write only "\cref{fig:example}" and the word "Figure" (or table, etc) will be inserted normally
\usepackage[noabbrev,capitalise]{cleveref}

\usepackage[
    maxbibnames=99,
    style=alphabetic,
    url=false,
    backend=bibtex8,
    sortcites=true,
]{biblatex}
\addbibresource{refs.bib}
\DeclareFieldFormat[online]{urldate}{Last accessed: #1}
\DeclareFieldFormat{eprint}{arXiv: \href{https://arxiv.org/abs/#1}{#1}}
\DeclareFieldFormat[report]{title}{``#1''}


\newcommand{\projectTitle}{Exploring Datastore Access Latency across AWS Compute Services}
\newcommand{\thesisType}{Bachelor's Thesis}
\newcommand{\authors}{Bhupendra Singh}
\newcommand{\matrikel}{464877}
\newcommand{\authorEmail}{\href{mailto:b.singh@campus.tu-berlin.de}{b.singh@campus.tu-berlin.de}}
\newcommand{\examinera}{Prof.~Dr.-Ing.~David Bermbach}
\newcommand{\examinerb}{Prof.~Dr.~habil.~Odej Kao}
\newcommand{\supervisor}{Trever Schirmer}

\newcommand{\projectYear}{2024}
\newcommand{\facultyName}{Fakultät Elektrotechnik und Informatik}
\newcommand{\departmentName}{Fachgebiet Scalable Software Systems}

\begin{document}
\input{front.tex}

\newpage

\section*{Abstract}
Cloud computing has grown rapidly, driving the demand for scalable and flexible infrastructure solutions. Amazon Web Services (AWS) leads the market, offering diverse compute and datastore services for various applications. Despite their importance, empirical data on latency dynamics between AWS compute and datastore services remains scarce.
%
This thesis addresses this knowledge gap by systematically benchmarking access latency between EC2 and Lambda paired with RDS, DynamoDB, and S3. We evaluate these service pairs under constant and burst workloads. The results show that EC2 pairs consistently achieve lower mean latency than Lambda pairs.
%
This study acknowledges certain limitations, including constraints imposed by AWS Free Tier, the focus on a single region, and the exclusion of mixed workload scenarios.
%
This work provides foundational insights for designing latency-sensitive cloud applications. It emphasizes the need for further research on multi-region setups, advanced configurations, and mixed workload scenarios to deepen the understanding of AWS service interactions.

% GERMAN
\clearpage
\begin{otherlanguage}
    {ngerman}
    \section*{Kurzfassung}
    Cloud Computing hat sich schnell entwickelt und führt zu einer steigenden Nachfrage nach skalierbaren und flexiblen Infrastruktur-Lösungen. Amazon Web Services (AWS) führt den Markt an und bietet eine Vielzahl von Compute- und Datenspeicher-Diensten für unterschiedlichste Anwendungen. Trotz ihrer Bedeutung fehlen jedoch empirische Daten zu den Latenz-Dynamiken zwischen den Compute- und Datenspeicher-Diensten von AWS.
    %
    Diese Arbeit schließt diese Wissenslücke, indem sie die Zugriffs-Latenz zwischen EC2 und Lambda in Kombination mit RDS, DynamoDB und S3 systematisch benchmarkt. Diese Dienst-Paare werden unter konstanten und burstigen Arbeitslasten evaluiert. Die Ergebnisse zeigen, dass EC2-Paare durchgehend eine niedrigere durchschnittliche Latenz aufweisen als Lambda-Paare.
    %
    Die Arbeit erkennt bestimmte Einschränkungen an, darunter die Begrenzungen durch das AWS Free Tier, die Fokussierung auf eine einzelne Region und die Ausschließung gemischter Arbeitslastszenarien.
    %
    Diese Arbeit liefert grundlegende Erkenntnisse für das Design latenzempfindlicher Cloud-Anwendungen und betont die Notwendigkeit weiterer Forschung zu Multi-Region-Setups, erweiterten Konfigurationen und gemischten Arbeitslastszenarien, um das Verständnis der Interaktionen zwischen AWS-Diensten zu vertiefen.
\end{otherlanguage}

\clearpage

\tableofcontents

\section{Introduction}
\label{cha:intro}

% BACKGROUND AND MOTIVATION
Cloud computing delivers on-demand resources like servers, storage, and databases via the Internet, eliminating the need for physical infrastructure and enabling flexibility and scalability. Its adoption has surged in the past decade due to the emergence of data-intensive applications and improved connectivity \cite{}. Amazon Web Services (AWS), a leader in cloud computing, pioneered various cloud models and controls over 30\% of the global market as of 2023, driven by its innovation and broad service portfolio \cite{}.

% PROBLEM STATEMENT
Applications in cloud environments typically consist of compute and datastore service instances that interact frequently. For example, AWS Elastic Compute Cloud (EC2) is often used for backend processing, while AWS Relational Database Service (RDS) handles structured data storage. Communication between these services is a critical aspect of cloud application architecture, and network latencies significantly influence overall performance and user experience, particularly in latency-sensitive applications such as e-commerce, real-time analytics, and gaming \cite{}.

AWS offers latency insights for datastore services from both service-side and client-side perspectives. However, it remains unclear how client-side latency varies across different AWS compute services. For instance, does the EC2-RDS pair perform differently compared to Lambda-RDS? AWS provides numerous configurations and tools for optimizing service interactions, yet the specific latency characteristics between services can vary depending on network setup, service proximity, caching strategies, and service-specific characteristics. As organizations seek efficient and performant cloud architectures, understanding the latency dynamics between these services is essential for making informed architectural decisions.

Despite the critical role of latency in cloud-based systems, publicly available empirical data on latency characteristics between AWS compute and datastore services remains limited. This thesis aims to fill this knowledge gap by providing an initial benchmark for access latency metrics between key AWS compute and datastore service pairs, specifically EC2, and Lambda paired with RDS, DynamoDB, and Simple Storage Service (S3).

% THESIS STRUCTURE
We therefore make the following contributions in this thesis:
\begin{enumerate}
	\item We address the identified knowledge gap by proposing a benchmark design, detailed in \cref{cha:approach}.
	\item We evaluate the proposed approach through experimentation on AWS compute and datastore service pairs, with implementation and results presented in \cref{cha:implementation,cha:results}.
\end{enumerate}

\clearpage
\section{Background}
\label{cha:background}

Access latency refers to the time required for data to travel between two points, often termed latency or Round-Trip Time (RTT). In the context of this thesis, it specifically refers to the time a compute service instance requires to obtain a response from the datastore. It is a well-known fact, that even small amounts of delay, even on a microsecond scale, may lead to a negative impact on the performance of an application \cite{atricle_dean_tail,book_popescu_netlat}, which implies the importance of latency.
%
From a cloud user's perspective, latency between cloud services can be influenced by certain controllable factors, such as service configuration, choice of region and availability zone, the proximity of resources, selecting instance types or service tiers optimized for low latency, network configuration, caching strategies, and application code optimization. However, underlying cloud infrastructure, especially network conditions remains uncontrollable.

Benchmarking carries different definitions across domains, however, in this thesis, it specifically refers to systematically evaluating and comparing access latency between AWS compute and datastore services. Cloud benchmarking is a well-researched field, therefore the following subsections summarize the key elements and requirements of cloud benchmarking, drawing from existing literature \cite{paper_binnig_weather,paper_cooper_ycsb,paper_folkerts_benchmarking,book_bermbach_cloud_service_benchmarking}. Additionally, we define relevant terms to provide clarity for subsequent sections.

\subsection{Elements of Cloud Benchmarking}
\label{elems_of_bench}

Cloud benchmarking typically involves a System Under Test (SUT) and a workload generator. The SUT may represent a single cloud service or an entire application containing components of interest. The workload generator applies an artificial load on the SUT while tracking the metric of interest. In this thesis, an AWS service pair consisting of a compute instance and a datastore constitutes the SUT. However, when the compute instance itself generates load on the datastore, the tool or executable responsible for load generation is considered the workload generator.

Workload generation is a critical element of cloud benchmarking, as it directly impacts SUT. Two common workload models are open and closed models. In a closed model, a fixed number of concurrent threads independently execute a predefined sequence of tasks iteratively. In contrast, the open model specifies a rate of arrival, such as request per second. The closed model has a fundamental limitation: it ties load generation to task completion. New tasks are only scheduled once a thread completes the previous one, allowing the SUT to self-regulate. If the SUT slows or stalls, incoming load decreases, enabling recovery and potentially skewed results.

A benchmarking run refers to a single-timed execution of a predefined set of tests or workloads to evaluate the performance of an SUT under specific conditions. It involves provisioning the resources, initiating the test, collecting relevant performance data, and storing the results for further analysis.

\subsection{Requirements of Cloud Benchmarking}

Cloud benchmarking requires several general considerations: \textbf{Relevance} ensures that the benchmarks reflect real-world scenarios and test conditions that align with the application's needs. \textbf{Repeatability} ensures that results can be consistently reproduced under similar conditions, enhancing the reliability of the findings. \textbf{Fairness} is vital to ensure that all SUTs are compared equitably, with identical configurations, resources, and network conditions to avoid bias. \textbf{Affordability} focuses on minimizing the costs of conducting the benchmark, particularly when using cloud resources with strict budget constraints. Additionally, \textbf{simplicity} is important for creating benchmarks that are easy to understand, interpret, and build trust.

To meet advanced cloud benchmarking requirements, benchmarks should account for \textbf{failure scenarios}, testing how services handle failure conditions. They must also consider the \textbf{geo-distribution} of both measurement clients and the SUT, reflecting real-world, geographically dispersed applications. Additionally, benchmarks should provoke \textbf{stress situations} to measure qualities like \textbf{scalability and elasticity}, using variable load patterns to test service limits. The design must ensure \textbf{detailed data capture}, avoiding reliance solely on aggregate values, and should track resource consumption and costs across various components using monitoring tools. Finally, benchmarks should be \textbf{long-running} and executed across different times of day and days of the week to capture stabilized behavior, short-term effects, and seasonal variations \cite{book_bermbach_cloud_service_benchmarking}.

Trade-offs between requirements are inevitable, as some are inherently conflicting. For example, prioritizing relevance may lead to complex applications that reduce simplicity, while long-running experiments can increase costs, affecting affordability. Therefore, it is essential to carefully consider all requirements and make conscious trade-offs based on the benchmarking goals and scope \cite{book_bermbach_cloud_service_benchmarking}.

\subsection{Terms and Definitions}
\label{challenges}

\paragraph{k6\footcite{https://k6.io/open-source/}}
k6 is an open-source and extensible workload generation tool. It uses JavaScript-based test scripts to define workloads, specifying parameters such as duration, arrival rate (or concurrent threads), load patterns, and operations on the SUT. It is written in the golang with embedded JavaScript engine\footcite{https://github.com/grafana/k6}.

\paragraph{Virtual Private Cloud (VPC)} VPC is an isolated network within the AWS cloud that allows users to launch AWS resources, such as EC2 instances, within a defined IP address range. It provides control over network configuration, including subnets, route tables, and security settings, enabling secure and scalable deployment of cloud resources.

%\paragraph{EC2}
%EC2 is a computing service that provides scalable virtual computing resources, allowing users to deploy and manage virtual machines (instances) with varying configurations of CPU, memory, storage, and networking.

\paragraph{Lambda}
Lambda is a serverless computing service that automatically runs code in response to events, eliminating the need to manage the underlying infrastructure. It scales automatically based on incoming requests, with users only charged for execution time. It introduces cold starts, where a new execution environment is initialized when the function is invoked after being idle, causing slight latency. To reduce this, provisioned concurrency keeps a predefined number of instances ready to handle requests, while reserved concurrency limits the maximum concurrent executions per function.

%\paragraph{RDS}
%RDS is a managed database service that supports various relational database engines, including MySQL, PostgreSQL, and Aurora.

\paragraph{DynamoDB}
DynamoDB is a fully managed NoSQL database service designed for low-latency, high-throughput applications. It supports the provisioned capacity mode, where users allocate Read Capacity Units (RCUs) and Write Capacity Units (WCUs) to define the number of reads and writes per second the database can handle. An RCU represents one strongly consistent read per second for an item up to 4 KB, while a WCU allows one write per second for an item up to 1 KB. 

%\paragraph{S3}
%Amazon S3 (Simple Storage Service) is a scalable, highly durable, and secure object storage service designed to store and retrieve any amount of data from anywhere.

\clearpage
\section{Benchmark Design}
\label{cha:approach}

To address the main purpose of this thesis, we adopt a benchmarking approach to evaluate the pairs and compare their results. This section presents the benchmark design, which is based on the guidelines outlined in \cite{}. As described in \cref{cha:intro}, the study examines six pairs of AWS compute and datastore services:

\begin{itemize}
	\item EC2 paired with RDS, DynamoDB, and S3
	\item Lambda paired with RDS, DynamoDB, and S3
\end{itemize}

Each benchmarking run consists of two main components: (1) a pre-loaded datastore serving as the System Under Test (SUT) and (2) a compute instance functioning as the workload generator. The compute instance performs read operations on the datastore and records latency metrics.

For Lambda-Pairs, each invocation corresponds to a single read operation on the datastore. To invoke the Lambda function in a scripted manner and collect results, an additional EC2 instance, referred to as the "Lambda-Helper", is provisioned with k6 and the necessary scripts, as shown in \ref{}. This instance sends HTTP requests to invoke the Lambda function and collects response data, which includes timestamps recorded by the Lambda function.

The benchmark evaluates each pair under two workload types, with each configuration repeated twice, resulting in of 24 runs. We use infrastructure-as-code and shell scripting for resource provisioning and de-provisioning in AWS, ensuring a high level of automation to minimize human error. All runs are executed in the same AWS region.

After each run is successfully completed, data is exported to a dedicated S3 bucket for collection. Finally, we transform the collected data and perform analysis to gain insights.

--> TODO: Add diagram here <--

\clearpage
\section{Benchmark Implementation}
\label{cha:implementation}

This section outlines the implementation details of the benchmark, focusing on three main aspects: (1) service instance configurations, (2) workload generation, and (3) data analysis.

For EC2-Pairs, the benchmarking process is straightforward, with the EC2 instance generating workload on datastores using k6 and storing the collected data locally. For Lambda-Pairs, each Lambda invocation corresponds to a single read operation on the datastore. To invoke the Lambda function programmatically and collect results, an additional EC2 instance, termed as "Lambda-Helper," is provisioned with k6 and the necessary test scripts, as shown in \ref{}.

All benchmarking runs are executed in the \textit{eu-central-1} AWS region. All components of a run are provisioned and de-provisioned using Infrastructure-as-Code  combined with shell scripting. This approach ensures a fresh environment for every run and minimizes the potential for human error. Upon successful completion of a run, the collected data is exported to a dedicated S3 bucket. Additionally, to keep network conditions and resource allocation identical, no benchmarking runs are executed simultaneously.

--> Diagram here <--

\subsection{Resource Configuration}
\label{sec:config}

\textbf{RDS}:
A MySQL 8.0 instance on a t3.micro virtual machine (VM) with 1GiB RAM, 2 vCPUs, and 5GB storage. The instance is located in the eu-central-1a availability zone, with multi-AZ failover disabled to maintain a single-zone configuration. It hosts a single database containing a table with 1000 rows and four columns.


\textbf{DynamoDB}:
A DynamoDB table with 10 items, read capacity of 25 Read Capacity Units (RCUs), write capacity of 5 Write Capacity Units (WCUs), and configured with \textit{Provisioned} capacity mode.

\textbf{S3}:
A single S3 bucket with a 20-byte text file.

\textbf{EC2}:
An EC2 instance of t2.micro type with 1GiB RAM, and 1vCPU, located in the eu-central-1a availability zone. It is operated by Ubuntu Server 24.04 (HVM-based) and is equipped with the k6 and corresponding test scripts.

\textbf{Lambda}:
A non-VPC Node.js 20 function with 1.65GiB RAM, 1vCPU, and a concurrency limit set to 990. The configuration of Lambda-Helper is identical to that of EC2.

\subsection{Workload Generation}
\label{sec:loads}

With following workload configuration we specifically address relevance, affordability, repeatability, and cloud performance fluctuations. However, for relevance, there is no standard rate or load pattern at which, for instance, a backend server reads from a datastore, and depends highly on the application's use-case and demand. Instead, we focus on extending the run duration to a point where it can be repeated twice while remaining within AWS Free Tier. Time limitations are also considered.

\textbf{Constant Workload}:
\begin{itemize}
	\item \textbf{RDS and DynamoDB}: 2 RPS for 14 hours.
	\item \textbf{S3}: 0.2 RPS (1 read every 5 seconds) for 3 hours.
\end{itemize}

\textbf{Burst Workload}:
\begin{itemize}
	\item \textbf{RDS and DynamoDB}: The run starts with a baseline of 2 RPS for the first 2 hours, followed by 30 load spikes. Each spike lasts 3 minutes, with RPS increasing linearly from 2 to 20 in 90 seconds and then decreasing back to the baseline of 2 in 90 seconds. Between spikes, the load remains at 2 RPS for 3 minutes. The run concludes with an additional 2 hours at the 2 RPS, resulting in a total test duration of approximately 7 hours.
	%
	\item \textbf{S3}: The run starts with a baseline of 0.2 RPS for the first 20 minutes, followed by 6 load spikes. Each spike lasts 30 seconds, with RPS increasing linearly from 0.2 to 16 in 15 seconds and then decreasing back to the baseline of 0.2 in 15 seconds. Between spikes, the load remains at 0.2 RPS for 6 minutes. The run concludes with an additional 15 minutes at the 0.2 RPS, resulting in a total test duration of approximately 70 minutes.
\end{itemize}

As noted, the monthly limits for S3 are relatively low, specifically 20000 GET requests per month, which resulted in shorter runs with S3.

\subsection{Data Analysis}
\label{sec:analysis}

As discussed, for each targeted datastore service, we compare latency performance across EC2 and Lambda. Mean values alone cannot fully capture the behavior of access latency throughout the experiment, so we use bar plots for aggregated metrics and time-series graphs to provide a comprehensive view of the data.

In our context, outliers in latency values may indicate potential network issues between services and can distort analysis, particularly when calculating averages \cite{book_bermbach_cloud_service_benchmarking}. To address this, we identify and remove outliers using the 3-sigma rule, which defines data points beyond three standard deviations from the mean as outliers \cite{}. However, the number and extent of these outliers provide valuable insights into a service pair's performance and reliability. Therefore, we also analyze and report outliers to understand their implications and highlight variations between pairs.





\clearpage
\section{Benchmark Results}
\label{cha:results}

In this section, we present and compare the findings from the data analysis across each targeted datastore. Throughout the benchmarking sessions, no CPU performance issues were detected on the client-side (load generator), ensuring that client limitations did not impact latency measurements. Additionally, the error rate consistently remained at zero for all runs, indicating that every read request was successfully completed.

\begin{figure}[h]
	\begin{subfigure}{0.49\linewidth}
		\centering
		\includegraphics[width=\linewidth]{./fig/bar-rds-constant.pdf}
		\caption{Constant Load on RDS}
		\label{fig:bar_rds_const}
	\end{subfigure}
	\hfill
	\begin{subfigure}{0.49\linewidth}
		\centering
		\includegraphics[width=\linewidth]{./fig/bar-rds-bursty.pdf}
		\caption{Bursty Load on RDS}
		\label{fig:bar_rds_bursty}
	\end{subfigure}
	\vfill
	\begin{subfigure}{0.49\linewidth}
		\centering
		\includegraphics[width=\linewidth]{./fig/bar-dynamo-constant.pdf}
		\caption{Constant Load on DynamoDB}
		\label{fig:bar_ddb_const}
	\end{subfigure}
	\hfill
	\begin{subfigure}{0.49\linewidth}
		\centering
		\includegraphics[width=\linewidth]{./fig/bar-dynamo-bursty.pdf}
		\caption{Bursty Load on DynamoDB}
		\label{fig:bar_ddb_bursty}
	\end{subfigure}
	\vfill
	\begin{subfigure}{0.49\linewidth}
		\centering
		\includegraphics[width=\linewidth]{./fig/bar-s3-constant.pdf}
		\caption{Constant Load on S3}
		\label{fig:bar_s3_const}
	\end{subfigure}
	\hfill
	\begin{subfigure}{0.49\linewidth}
		\centering
		\includegraphics[width=\linewidth]{./fig/bar-s3-bursty.pdf}
		\caption{Bursty Load on S3}
		\label{fig:bar_s3_bursty}
	\end{subfigure}
	\caption{Aggregation of Latency Metric}
	\label{fig:bar-plots}
\end{figure}

\begin{figure}[h]
	\begin{subfigure}{0.49\linewidth}
		\centering
		\includegraphics[width=\linewidth]{./fig/ts-rds-constant.pdf}
		\caption{Constant Load on RDS}
		\label{fig:ts_rds_const}
	\end{subfigure}
	\hfill
	\begin{subfigure}{0.49\linewidth}
		\centering
		\includegraphics[width=\linewidth]{./fig/ts-rds-bursty.pdf}
		\caption{Bursty Load on RDS}
		\label{fig:ts_rds_bursty}
	\end{subfigure}
	\vfill
	\begin{subfigure}{0.49\linewidth}
		\centering
		\includegraphics[width=\linewidth]{./fig/ts-dynamo-constant.pdf}
		\caption{Constant Load on DynamoDB}
		\label{fig:ts_ddb_const}
	\end{subfigure}
	\hfill
	\begin{subfigure}{0.49\linewidth}
		\centering
		\includegraphics[width=\linewidth]{./fig/ts-dynamo-bursty.pdf}
		\caption{Bursty Load on DynamoDB}
		\label{fig:ts_ddb_bursty}
	\end{subfigure}
	\vfill
	\begin{subfigure}{0.49\linewidth}
		\centering
		\includegraphics[width=\linewidth]{./fig/ts-s3-constant.pdf}
		\caption{Constant Load on S3}
		\label{fig:ts_s3_const}
	\end{subfigure}
	\hfill
	\begin{subfigure}{0.49\linewidth}
		\centering
		\includegraphics[width=\linewidth]{./fig/ts-s3-bursty.pdf}
		\caption{Bursty Load on S3}
		\label{fig:ts_s3_bursty}
	\end{subfigure}
	\caption{Time-Series Representation of Latency Metric}
	\label{fig:ts-plots}
\end{figure}

As observed in \cref{fig:bar-plots}, mean latencies of EC2-Pairs is less than that of Lambda-Pairs for all experiments. The absolute difference is although minimal, always less than 1.34 milliseconds. 

In \hyperref[fig:bar_s3_const]{Figure 4.1.5}, we can obeserve that EC2-S3 under constant load shows approximately 3x variance compared to EC2-S3 under bursty load in \hyperref[fig:bar_s3_bursty]{Figure 4.1.6}, which intuitively seems contradicting. Similar is observed in \cref{fig:ts-plots}, especially in \hyperref[fig:ts_rds_const]{Figure 4.2.1}, where latency with lambda suddenly rises or sinks and continues at that rate for multiple hours. Possible reason is that the underlying conditions of the cloud vary depending on day and time, leading to variations in latency performance \cite{}.

The reason behind EC2 outperforming Lambda possibly lies in the fact that, Lambdas reside inside Firecracker-VMs, which are deployed on multi-tenant EC2s. The Lambdas communicate with the host EC2s via istio-based network interface. This introduces networking overhead which might impact latency performance \cite{}.

\paragraph*{EC2-RDS vs. Lambda-RDS}

\paragraph*{EC2-DynamoDB vs. Lambda-DynamoDB}

\paragraph*{EC2-S3 vs. Lambda-S3}

\paragraph*{Outliers}

\clearpage
\section{Discussion}
\label{cha:discuss}

\cref{cha:results} shows that datastore access latency is influenced by the choice of compute service, with EC2 consistently outperforming Lambda across all experiments. This suggests an inherent variance and overhead associated with Lambda. In this section, we explore possible reasons for these observations and also assess our approach and its limitations.

In \cref{fig:ts-plots}, we observe temporal shifts in the latency performance, even under constant workload. Temporal performance variability in serverless platforms, including AWS Lambda, is well-documented \cite{paper_ginzburg_lambda_var,article_eismann_lambda_var,paper_schirmer_night_shift}. Ginzburg et al. \cite{paper_ginzburg_lambda_var} report significant performance shifts in AWS Lambda during a one-week continuous benchmark, even suggesting that these variations could be exploited to reduce costs. It is possible that our benchmarking runs, specifically Lambda-RDS (\hyperref[fig:bar_rds_const]{Figure 5.2.1} and \hyperref[fig:bar_rds_bursty]{5.2.2}) and Lambda-Dynamo (\hyperref[fig:bar_ddb_const]{Figure 5.2.3} and \hyperref[fig:bar_ddb_bursty]{ 5.2.4}), were affected by such temporal shifts, impacting the results. Similar is also applicable for EC2 \cite{paper_iosup_performance,article_schad_cloud_var}, however, Dancheva et al. \cite{article_dancheva_ec2_var} suggest that recent advances are able to improve networking performance in EC2.

Lambda functions run inside microVMs based on Firecracker\footcite{https://firecracker-microvm.github.io/} hypervisor, deployed on bare-metal, multi-tenant EC2 instances referred as Lambda-Workers. Unlike standard EC2 instances based on the AWS Nitro\footcite{https://aws.amazon.com/ec2/nitro/} hypervisor, Lambda-Workers add Firecracker as an additional abstraction layer. Communication between a microVM and Firecracker occurs via optimized virtio\footcite{https://libvirt.org/} interface \cite{paper_brooker_lambda}, leading to overall networking overhead of approximately $0.06ms$ under controlled, isolated conditions \cite{repo_firecracker}. In production, however, Lambda-Workers may host hundreds or thousands of microVMs \cite{paper_agache_firecracker}, potentially increasing this overhead due to induced network noise and resource contention \cite{article_desensi_noise,paper_wang_faas_bts}. Agache et al. \cite{paper_agache_firecracker} also highlight Firecracker's comparatively lower networking performance against other hypervisors. Furthermore, Lambda-Workers are hosted within a network-isolated VPC managed by Lambda in the service accounts inaccessible to customers \cite{web_aws_lambda_security}, leaving their precise network configuration unclear.

We observe a minimum latency of $0ms$ in \hyperref[fig:bar_rds_const]{Figure 5.1.1} and \hyperref[fig:bar_rds_bursty]{5.2.2}, which is due to the $1ms$ measurement accuracy, causing latencies below $1ms$ to be rounded to $0ms$. As the differences are small, on the microsecond level, high-resolution timestamps with $1\mu s$ accuracy are better suited for more precise results.

\paragraph*{Limitations} In \cref{cha:background}, we outline general aspects and requirements of cloud benchmarking, present our approach in \cref{cha:approach}, and validate it through experimentation. However, affordability emerges as a significant limitation. AWS Free Tier constraints restrict experiment duration and scale, limiting long-term performance trend observations. The use of low-tier configurations, such as t2.micro instances and no provisioned concurrency for Lambda, hinder evaluation under realistic production conditions. Temporal variability in cloud performance poses another challenge, with short-duration benchmarks, particularly for S3, possibly failing to capture the real empirical distribution of access latency. The absence of multi-region setups excludes the analysis of geo-distributed latency behaviors.

\clearpage
\section{Related Work}
\label{cha:relatedwork}

--> TODO <--
\clearpage
\section{Conclusion}
\label{cha:conclusion}

This thesis systematically examined datastore access latency across AWS compute services, specifically focusing on EC2 and Lambda paired with RDS, DynamoDB, and S3, under two workload types. The results indicate that EC2 consistently outperforms Lambda in terms of mean latency, however with minor differences. Lambda-based pairs exhibited greater latency susceptibility to performance shifts, likely due to inherent architectural overheads and variability. These findings emphasize EC2's stability for latency-sensitive applications while highlighting the trade-offs of Lambda's serverless model.

The study also faced limitations, including constraints of the AWS Free Tier, short duration, the use of simplified configurations, and single-region setups. These factors underscore the need for future research to incorporate multi-region benchmarks, advanced configurations, and mixed workload scenarios for a deeper understanding of cloud service interactions. The insights gained from this work provide a foundation for cloud architects and application developers to optimize AWS service selections, balancing performance, scalability, and cost-efficiency.



\newpage
\printbibliography
\end{document}
