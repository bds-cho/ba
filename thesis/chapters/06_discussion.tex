\section{Discussion}
\label{cha:discuss}

This study evaluated latency performance between AWS compute services (EC2 and Lambda) and three datastore options (RDS, DynamoDB, and S3). The findings indicate latency trends influenced by compute-service architecture and workload characteristics. However, several aspects of the methodology and limitations must be critically analyzed to contextualize these outcomes.

This study encountered several significant limitations. The constraints of the AWS Free Tier restricted the duration and scale of experiments, limiting the observation of long-term performance trends. The reliance on low-tier configurations, such as t3.micro instances, prevented evaluation in realistic production environments where higher-capacity instances are standard. Temporal variability in cloud performance was another challenge, with short-duration benchmarks potentially not capturing the real distribution of the latency metric caused by time-of-day effects or transient network conditions. Additionally, the absence of multi-region setups meant geo-distributed latency behaviors were not analyzed. Simplified use cases focusing on read operations excluded mixed read-write workloads and asynchronous interactions, which are typical in production systems. Lastly, while statistical methods effectively identified and excluded outliers, their underlying causes were not investigated, leaving critical insights into rare performance anomalies unaddressed.

The study's findings have important implications for designing cloud-based systems. EC2 consistently demonstrated lower latency, making it a reliable choice for applications with stringent performance requirements. However, the relatively small differences in latency observed between Lambda and EC2, particularly with DynamoDB, indicate that Lambda could be a viable option for scenarios where operational simplicity and scalability are prioritized over minimal latency. To build upon these findings, future research should consider more extensive benchmarks involving realistic production configurations, mixed workloads, and multi-region setups to better understand the dynamics of latency in diverse scenarios.
