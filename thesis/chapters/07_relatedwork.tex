\section{Related Work}
\label{cha:relatedwork}

Cloud computing is a widely adopted paradigm, resulting in extensive literature exploring various aspects of its ecosystem. Given the scale of cloud applications and architectures, this study falls under the category of micro-benchmarking, where existing work remains relatively limited. Nonetheless, there are overlaps with studies focusing on different but related aspects of cloud performance.

Villamizar et al. \cite{paper_villamizar_lambdaXec2} compare infrastructure costs and performance of web applications deployed using three architectures: Lambda-based microservices, EC2-based microservices, and monolithic. For data storage, they use PostgreSQL on EC2, though it is unclear if RDS is employed. Unlike this thesis, their study measures latency from a web user’s perspective, factoring in variables like gateways, cold starts, load balancing, and internal communication overhead. Their findings indicate that the Lambda-based microservice architecture achieves better average response times for end users.

Klimovic et al. \cite{paper_klimovic_lambdaXs3} evaluate the suitability of storage options for serverless analytics, including S3, Redis\footcite{https://redis.io/}, and Crail-ReFlex\footcite{https://craillabs.github.io/}, to address ephemeral data sharing requirements. They highlight the overhead of using S3 for latency-sensitive serverless applications, identifying Redis as a lower-latency alternative. Their study examines latency between AWS Lambda and S3 for both read and write operations, reporting $12.1ms$ read and $25.8ms$ write latency for $1KB$ requests. Additionally, they analyze aspects like I/O time, throughput, job runtime, and concurrency.

Palepu et al. \cite{paper_palepu_lambdaXs3ddb} benchmark data transfer rates across serverless and datastore services, including AWS Lambda, S3, and DynamoDB. They identify factors affecting performance, such as memory allocation and concurrency, and highlight Lambda's limitations in scaling data transfer rates under high concurrency. The study reports that Lambda exhibits lower throughput with S3 and DynamoDB compared to EC2 but does not provide specific latency insights.

Albuquerque-Junior et al. \cite{paper_albu_lambdaec2} benchmark AWS Lambda against AWS Beanstalk \footcite{https://aws.amazon.com/elasticbeanstalk/}, which is based on EC2 instances, to evaluate performance in microservice architectures. They use PostgreSQL for data persistence, but it is unclear whether it is RDS. Unlike our approach, they focus on end-to-end performance using a closed workload model, with experiments lasting just over 4 minutes and an unspecified number of repetitions. There are fairness concerns due to the resource configuration: Lambda is allocated $256MB$ of memory ($0.167vCPU$), while Beanstalk uses a t1.micro EC2 instance with $1vCPU$ and $0.613GB$ of memory. They also examine the impact of resource allocation and service configuration. Their results show that EC2 with Beanstalk outperforms Lambda in terms of both read and write latencies.
