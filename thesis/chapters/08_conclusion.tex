\section{Conclusion}
\label{cha:conclusion}

This thesis systematically examined datastore access latency across AWS compute services, specifically focusing on EC2 and Lambda paired with RDS, DynamoDB, and S3, under two workload types. The results indicate that EC2 consistently outperforms Lambda in terms of mean latency, however with minor differences. Lambda-based pairs exhibited greater latency susceptibility to performance shifts, likely due to inherent architectural overheads and variability. These findings emphasize EC2's stability for latency-sensitive applications while highlighting the trade-offs of Lambda's serverless model.

The study also faced limitations, including constraints of the AWS Free Tier, short duration, the use of simplified configurations, and single-region setups. These factors underscore the need for future research to incorporate multi-region benchmarks, advanced configurations, and mixed workload scenarios for a deeper understanding of cloud service interactions. The insights gained from this work provide a foundation for cloud architects and application developers to optimize AWS service selections, balancing performance, scalability, and cost-efficiency.

